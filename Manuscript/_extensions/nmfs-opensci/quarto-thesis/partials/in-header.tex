\usepackage[utf8]{inputenc} % Required for inputting international characters
%\usepackage[T1]{fontenc} % Output font encoding for international characters; causes problems for xelatex

%\usepackage{mathpazo} % Use the Palatino font by default

\usepackage[backend=bibtex, style=authoryear, natbib=true]{biblatex} % Use the bibtex backend with the authoryear citation style (which resembles APA)

\usepackage[autostyle=true]{csquotes} % Required to generate language-dependent quotes in the bibliography

\usepackage{setspace}

\usepackage{tikz}
\usetikzlibrary{calc}
\usetikzlibrary{calc,decorations.pathmorphing}
\usepackage{tocloft}
\usepackage{lscape} % For landscape pages
\usepackage{geometry} % For custom page layout
\usepackage{graphicx} % For including images
\usepackage{caption} % For custom captions
\usepackage{truncate}
\usepackage{lipsum} % For dummy text
\usepackage{xcolor} % To define and use colors
\usepackage{eso-pic} % For background images % For including images
\definecolor{textgray}{HTML}{595959}
\definecolor{greenTriangle}{HTML}{1f7549}
\definecolor{greenText}{HTML}{2DA86A}
\usepackage{tikz} % Load the TikZ package
\usepackage{tabularx}
\usepackage{fontspec} % To use system fonts like Arial
\PassOptionsToPackage{scaled}{helvet}
\usepackage{helvet}
\usetikzlibrary{patterns,patterns.meta} % LaTeX and plain TeX when using TikZ
\usepackage{titlesec}
\titleformat{\paragraph}[runin]{\normalfont\normalsize\bfseries}{\theparagraph}{1em}{}
\usepackage{multicol}
%----------------------------------------------------------------------------------------
%	MARGINS
%----------------------------------------------------------------------------------------

\geometry{
	headheight=4ex,
	includehead,
	includefoot
}

\raggedbottom

\AtBeginDocument{
\hypersetup{pdftitle=\ttitle} % Set the PDF's title to your title
\hypersetup{pdfauthor=\authorname} % Set the PDF's author to your name
\hypersetup{pdfkeywords=\keywordnames} % Set the PDF's keywords to your keywords
}

\usepackage{tcolorbox}
\tcbset{
  colback=gray!10, % Background color (10% grey)
  colframe=gray!50, % Border color (50% grey)
  boxrule=0.5mm, % Border thickness
  arc=4mm, % Border radius
  left=1mm, % Left padding
  right=1mm, % Right padding
  top=1mm, % Top padding
  bottom=1mm, % Bottom padding
}

\renewcommand{\floatpagefraction}{0.8} % Allow floats to occupy up to 80% of a page
\renewcommand{\textfraction}{0.1}     % Allow text to occupy as little as 10% of a page
\renewcommand{\topfraction}{0.9}      % Allow up to 90% of the top of a page for floats
\renewcommand{\bottomfraction}{0.9}   % Allow up to 90% of the bottom of a page for floats

\newcommand{\startonrightwithgap}{%
  \clearpage
  % Check whether the current page is odd
  \ifodd\value{page}
    % -- We are on odd page X, so skip two pages (X+1, X+2):
    \thispagestyle{empty}\mbox{}\clearpage
    \thispagestyle{empty}\mbox{}\clearpage
  \else
    % -- We are on even page Y, so skip one page (Y+1):
    \thispagestyle{empty}\mbox{}\clearpage
  \fi
}

\newcommand{\startonleftwithgap}{%
  \clearpage
  \ifodd\value{page}
    % If the current page is odd, skip 3 blank pages
    \thispagestyle{empty}\mbox{}\clearpage
    \thispagestyle{empty}\mbox{}\clearpage
    \thispagestyle{empty}\mbox{}\clearpage
  \else
    % If the current page is even, skip 2 blank pages
    \thispagestyle{empty}\mbox{}\clearpage
    \thispagestyle{empty}\mbox{}\clearpage
  \fi
}
% 
% \newcommand{\startonrightwithgap}{%
%   \clearpage % End the current section cleanly
%   \ifodd\value{page} % Check if the current page is odd
%     \null\newpage % Add a blank page if odd
%     \thispagestyle{empty} % Ensure the blank page has no headers/footers
%   \else
%     \null\newpage % Add a blank page if even (this ensures X+1 is blank)
%     \thispagestyle{empty} % Ensure the blank page has no headers/footers
%     \null\newpage % Add another blank page for X+2
%     \thispagestyle{empty} % Ensure this blank page has no headers/footers
%   \fi
%   \cleardoublepage % Ensure the next section starts on a right (odd) page
% }